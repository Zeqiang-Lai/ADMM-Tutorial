\documentclass[10pt]{article}
\usepackage{minted, ctex, url}
\usepackage{amsthm, amsmath, amssymb, amsfonts}
\usepackage{graphicx, float, xcolor, subcaption}
%\usepackage[margin=1in]{geometry}
\usepackage{hyperref}
\usepackage[a4paper,left=3cm,right=3cm,top=2.5cm,bottom=2.5cm]{geometry}
\usepackage[toc,page]{appendix}
\usepackage{rotating}
\usepackage[hyperref=true,backend=biber,sorting=none,backref=true]{biblatex}
\addbibresource{ref.bib}

\title{ADMM Tutorial}
\author{赖泽强}
\newtheorem{theorem}{\hspace{2em}Theorem}[section]
\newtheorem{definition}{\hspace{2em}Definition}[section]
\newtheorem{lemma}{\hspace{2em}Lemma}[section]
\newtheorem{proposition}{\hspace{2em}Proposition}[section]
\newtheorem{corollary}{\hspace{2em}Corollary}[section]
\newcommand{\todo}[1]{{\color{red}[TODO: #1]}}

\newcommand{\F}{\mathbb F}
\newcommand{\E}{\mathbb E}
\newcommand{\R}{\mathbb R}
\newcommand{\Q}{\mathbb Q}
\newcommand{\Z}{\mathbb Z}
\newcommand{\N}{\mathbb N}
\newcommand{\Prob}{\mathbb P}

\newcommand{\textand}{\quad \text{and} \quad}
\newcommand{\textor}{\quad \text{or} \quad}

\DeclareMathOperator*{\argmax}{arg\,max}
\DeclareMathOperator*{\argmin}{arg\,min}
\newcommand{\norm}[1]{\left \Vert #1 \right \Vert}
\newcommand{\set}[1]{\left \{ #1 \right \}}
\newcommand{\tr}{\text{tr}} % trace
\newcommand{\Var}{\text{Var}} 
\newcommand{\diag}{\text{diag}} 
\newcommand{\inner}[2]{\left \langle #1, #2 \right \rangle}




\begin{document}
\maketitle
\tableofcontents

\section{Basic}

ADMM是针对优化问题的一种解法。优化问题则是说,我们希望在给定一些约束的情况下,去寻找一个解来最小化一个我们定义的目标函数。这个过程可以形式化地定义为:

$$
P: \min _{x \in D} f(x)
$$

其中f是目标函数,D是由约束条件划定的x的取值范围。

ADMM尝试解决的则是一种特殊的优化问题。在这个问题中,我们的目标函数是一个均方误差,约束条件则是一个L1范数。我们使用朗格朗日法将有约束问题转化为无约束问题,就变成了公式\ref{eq-1}所示的形式。

\begin{equation}
\min _{x} \frac{1}{2} \sum_{i=1}^{n}\left(y_{i}-x_{i}\right)^{2}+\lambda \sum_{(i, j) \in E}\left|x_{i}-x_{j}\right|
\label{eq-1}
\end{equation}


通常我们将这个问题称为 \textbf{2d fused lasso} 或 \textbf{2d total variation denoising} 问题


我们可以使用各种各样的优化算法来解这个特殊的问题,例如Proximal gradient descent,Coordinate descent,但ADMM是这些算法中收敛最快的(即需要迭代次数少)\footnote{需要注意的是,ADMM在这个问题上快,不代表它在其它问题上也快。}。

\subsection{Vanilla ADMM}

那么ADMM是怎么做的呢?ADMM先是引入了一个新的变量v,并约束$x=v$,然后解公式\ref{eq-1}所示的无约束问题,就变成了解公式\ref{eq-2}所示的有约束问题。

\begin{equation}
(\widehat{\boldsymbol{x}}, \widehat{\boldsymbol{v}})=\underset{\boldsymbol{x}, \boldsymbol{v}}{\operatorname{argmin}} \quad f(\boldsymbol{x})+\lambda g(\boldsymbol{v}), \quad \text { subject to } \boldsymbol{x}=\boldsymbol{v}
\label{eq-2}
\end{equation}

然后我们用增广朗格朗日法再将其变成无约束问题,变成公式\ref{eq-3}的形式\footnote{为什么要变成这种形式?一个直观的解释是新的函数更凸,而凸函数在优化是具有很好的性质,如收敛快,更容易获得更优解等。参见:交替方向乘子法(ADMM)算法的流程和原理是怎样的? - 大大大的v的回答 - 知乎 \url{https://www.zhihu.com/question/36566112/answer/118715721}}。

\begin{equation}
\mathcal{L}(\boldsymbol{x}, \boldsymbol{v}, \boldsymbol{u})=f(\boldsymbol{x})+\lambda g(\boldsymbol{v})+\boldsymbol{u}^{T}(\boldsymbol{x}-\boldsymbol{v})+\frac{\rho}{2}\|\boldsymbol{x}-\boldsymbol{v}\|^{2}
\label{eq-3}	
\end{equation}

优化这个方程可以使用分步优化的方法,即先选取一个优化变量,然后固定其它变量,对刚刚选取的变量进行优化,依次选取所有需要优化的变量重复进行。这个过程可以用公式\ref{eq-4}描述。

\begin{equation}
\begin{array}{l}
\boldsymbol{x}^{(k+1)}=\underset{\boldsymbol{x} \in \mathbb{R}^{n}}{\operatorname{argmin}} \quad f(\boldsymbol{x})+\frac{\rho}{2}\left\|\boldsymbol{x}-\tilde{\boldsymbol{x}}^{(k)}\right\|^{2} \\
\boldsymbol{v}^{(k+1)}=\underset{\boldsymbol{v} \in \mathbb{R}^{n}}{\operatorname{argmin}} \quad \lambda g(\boldsymbol{v})+\frac{\rho}{2}\left\|\boldsymbol{v}-\widetilde{\boldsymbol{v}}^{(k)}\right\|^{2} \\
\overline{\boldsymbol{u}}^{(k+1)}=\overline{\boldsymbol{u}}^{(k)}+\left(\boldsymbol{x}^{(k+1)}-\boldsymbol{v}^{(k+1)}\right)
\end{array}
\label{eq-4}
\end{equation}

其中第三个优化$u$的式子,我们是要让u最大,用这种方式强迫x和v更接近。然后因为我们求导有解析解,我们可以直接使用梯度上升法。

至于ADMM这种做法为什么会获得更快的收敛速度,我还没有深入研究。

\subsection{Plug-and-Play ADMM}

对于公式\ref{eq-4}里的第二个式子,定义$\sigma = \sqrt{\lambda \rho}$,我们可以将其改写成:

\begin{equation}
\boldsymbol{v}^{(k+1)}=\underset{\boldsymbol{v} \in \mathbb{R}^{n}}{\operatorname{argmin}} \quad g(\boldsymbol{v})+\frac{1}{2 \sigma^{2}}\left\|\boldsymbol{v}-\tilde{\boldsymbol{v}}^{(k)}\right\|^{2}
\label{admm-v}
\end{equation}

直观的,我们可以把这个优化过程看出一个降噪的过程,其中$\sigma$是高斯噪声的强度(我们假设噪声是高斯噪声)。我们可以把v当成降噪后的图像,$v^{k}$看出带噪声的图像。$g(v)$是说我们降噪后的图像要是一个图像(满足先验g),后面一项则是说降噪后的图像和原图像要接近。

因此,我们可以使用一个降噪器去替代这个优化过程。每一步优化,我们都将$v^{k}$输入一个降噪器获得$v^{k+1}$。

具体为什么说这个形式很像降噪,我们需要先回顾一下降噪的优化算法是什么样的。

对于一个降噪问题,我们形式化为如下的优化问题:

\begin{equation}
\widehat{\boldsymbol{x}}_{\operatorname{map}}=\arg \max _{\boldsymbol{x}} p(\boldsymbol{x} \mid \boldsymbol{y})
\end{equation}

其中y是输入的噪声图像,x是去噪图像。

使用贝叶斯公式,加负号,我们可以将其转换成如下的优化问题:

\begin{equation}
\widehat{\boldsymbol{x}}_{\operatorname{map}}=\arg \min _{\boldsymbol{x}}\{-\ln p(\boldsymbol{y} \mid \boldsymbol{x})-\ln p(\boldsymbol{x})\}
\label{denoise}
\end{equation}

如果我们假设噪声是高斯噪声,那么$e=y-x$应该服从正态分布,即$\boldsymbol{e} \sim \mathcal{N}\left(\mathbf{0}, \sigma_{e}^{2} \boldsymbol{I}\right)$。

因为$p(y|x)$是给定原始图像,噪声图像出现的概率,既然我们知道噪声的概率分布,那$p(y|x)$事实上应该就是噪声出现的概率。因此我们可以将正态分布的公式代入,求解出$-ln p(y|x)$:

$$- logP(y|x) = -log(\frac{1}{\sigma\sqrt{2\pi}}e^{-\frac{(y-x)^2}{2\sigma}})=\frac{(y-x)^2}{2\sigma}+log(\sigma\sqrt{2\pi})\propto \frac{(y-x)^2}{2\sigma}$$

到这里,就不难看出为什么我们说ADMM优化v的步骤可以看成一个降噪过程了,对比公式\ref{admm-v}和公式\ref{denoise},前者的g(v)就相当于后者的$-lnp(x)$,前者的后一个平方差项和后者则是完全一致的。

\subsection{Tuning Free PnP ADMM}

在PnP ADMM中,观察公式\ref{eq-4},我们知道这个算法存在两个超参数$\rho$和$\sigma$。Tuning Free PnP ADMM\cite{wei2020tuning}这个算法就是使用强化学习的方法去自动寻找\textbf{每一步迭代}最适合的参数。

因为每一步迭代都可以使用不同的超参数,因此有时候可以获得比人工调参,甚至穷举\footnote{穷举是指在最开始穷举得到一个最优参数,但每一步迭代参数相同,因为每一步迭代都穷举并不现实。}更优的结果。

当然,这个算法最大的好处还是不用自动化了调参的过程。

\section{Examples}

\subsection{ADMM 1D TV Denosing}

1D TV Denosing问题描述如下,其中F是一个Difference matrix,主对角线全是1,主对角线上方元素是-1。

\begin{equation}
	\operatorname{minimize} \enspace \frac{1}{2} \|x-y\|_{2}^{2}+\lambda\|F x\|_{1}
\end{equation}

ADMM形式:

\begin{equation}
\begin{array}{ll}
\operatorname{minimize} & \frac{1}{2} \|x-y\|_{2}^{2}+\lambda\|z\|_{1} \\
\text { subject to } & F x-z=0
\end{array}
\end{equation}

增广朗格朗日形式:
\begin{equation}
L_{\rho}( x , z , \nu )= \frac{1}{2}\|x-y\|_{2}^{2} + \lambda\|z\|_{1}+ \nu ^{ T }(Fx-z)+\frac{\rho}{2}\|Fx-z\|_{2}^{2}
\label{eq-3}
\end{equation}

令$\mu = \nu / \rho$,可以验证:
\begin{equation}
\nu ^{ T }(Fx-z)+\frac{\rho}{2}\| Fx-z\|_{2}^{2}= \frac{\rho}{2}\|Fx-z+ \mu \|_{2}^{2}-\frac{\rho}{2}\| \mu \|_{2}^{2}
\end{equation}

因此,新的增广朗格朗日形式可以写成:

\begin{equation}
L_{\rho}( x , z , \nu )= \frac{1}{2}\|x-y\|_{2}^{2} + \lambda\|z\|_{1}+ \frac{\rho}{2}\|Fx-z+ \mu \|_{2}^{2}-\frac{\rho}{2}\| \mu \|_{2}^{2}
\end{equation}

因此,ADMM的分布优化步骤为:

\begin{equation}
\begin{aligned}
x ^{(k+1)} &=\arg \min _{ x }\left(\frac{1}{2} \|x-y\|_{2}^{2}+\frac{\rho}{2}\left\| Fx-z ^{(k)}+ \mu ^{(k)}\right\|_{2}^{2}\right) \\
z ^{(k+1)} &=\arg \min _{ z }\left(\lambda\|z\|_{1}+\frac{\rho}{2}\left\| Fx ^{(k+1)}- z + \mu ^{(k)}\right\|_{2}^{2}\right) \\
\nu ^{(k+1)} &= \nu ^{(k)}+ Fx ^{(k+1)}- z ^{(k+1)}
\end{aligned}
\label{eq-6}
\end{equation}

这三个优化步骤都有解析解,如下(牢记:$\mu = \nu / \rho$):

\begin{equation}
\begin{aligned}
x^{k+1} &:=\left(I+\rho F^{T} F\right)^{-1}\left(y+\rho F^{T}\left(z^{k}-\mu^{k}\right)\right) \\
z^{k+1} &:=T_{\lambda / \rho}\left(F x^{k+1}+\mu^{k}\right) \\
\nu^{k+1} &:=\nu^{k}+F x^{k+1}-z^{k+1}
\end{aligned}
\end{equation}

具体含义和推导见下:

\paragraph{求解x: }

优化x等价于求解一个最小二乘问题。推导如下:

最小二乘法的目标函数为:

\begin{equation}
L(D, \vec{\beta})=\|X \vec{\beta}-Y\|^{2}
\end{equation}

有解析解:

\begin{equation}
	(X^TX)^{-1}X^TY
\end{equation}

改写公式\ref{eq-6}中x的目标函数:

\begin{equation}
	x ^{(k+1)} =\arg \min _{ x }\left(\|x-y\|_{2}^{2}+ \left\|\sqrt{\rho}Fx- \sqrt{\rho}(z ^{(k)}- \mu ^{(k)})\right\|_{2}^{2}\right)
\end{equation}

我们把这两个二范数合起来写成矩阵形式:

\begin{equation}
\min _{x}\left\|\left[\begin{array}{c} 
I \\
\sqrt{\rho} F
\end{array}\right] x -\left[\begin{array}{c} 
y \\
\sqrt{\rho}\left( z ^{(k)}- \mu ^{(k)}\right)
\end{array}\right]\right\|_{2}^{2}
\end{equation}

把x当成$\beta$, 把式子前面的矩阵当成$X$,把后面的矩阵当成$Y$,可知,优化x等价于一个最小二乘问题。

代入最小二乘法的解析解公式:

\begin{equation}
\begin{aligned}
x ^{(k+1)} &=\left( I +\rho F^TF \right)^{-1}\left[ I, \sqrt{\rho} F^T \right]\left[\begin{array}{c} 
y \\
\left.\sqrt{\rho}\left( z ^{(k)}- \mu ^{(k)}\right)\right]
\end{array}\right] \\
&=\left( I +\rho F^TF \right)^{-1}\left( y +\rho F^T\left( z ^{(k)}- \mu ^{(k)}\right)\right)
\end{aligned}
\end{equation}

\paragraph{求解z: }

z是一个一维向量,将z的目标函数展开,令$v=Fx+\mu$, 我们可以得到:

\begin{equation}
\begin{aligned}
& \underset{z}{\operatorname{minimize}} \enspace \lambda \sum_{n=1}^{N}|z[n]|+\frac{\rho}{2} \sum_{n=1}^{N}(z[n]-v[n])^{2} \\
=& \operatorname{minimize}_{ z } \enspace \sum_{n=1}^{N}\left(\lambda|z[n]|+\frac{\rho}{2}(z[n]-v[n])^{2}\right)
\end{aligned}
\end{equation}


因为z的每一个分量没有关系,我们可以单独优化z的每一个分量:

\begin{equation}
\underset{z \in R }{\operatorname{minimize}} \enspace \lambda|z|+\frac{\rho}{2}(z-v)^{2}
\end{equation}

这个函数除了0处处可导,且是个凸函数,导数为

\begin{equation}
\frac{ d f}{ d z}=\left\{\begin{array}{ll}
\lambda+z-\rho v, & z>0 \\
-\lambda+z-\rho v, & z<0
\end{array}\right.
\end{equation}

凸函数的极值点就是最值点,因此z的最优解为导数为0的地方。当$|v| > \lambda/\rho$时,导数可以取到0;反之,z等于0时取到导数绝对值最小的位置,即最优解。

\begin{equation}
z^{\star}=\left\{\begin{array}{ll}
\rho v-\lambda, & v>\lambda/\rho \\
0, & |v| \leq \lambda/\rho \\
\rho v+\lambda, & v<-\lambda/\rho
\end{array}\right.
\end{equation}

用$T_\lambda(\cdot)$表示这个函数,则

\begin{equation}
	z^{(k+1)} = T_{\lambda/\rho}(v) = T_{\lambda/\rho}(Fx+\mu)
\end{equation}

$T_\lambda(\cdot)$被称为\textbf{soft thresholding} 或 \textbf{shrinkage} operator。

\paragraph{求解$\nu$: } 使用公式\ref{eq-3}和求导即可:

\begin{equation}
\frac{d\nu}{dL} = (Fx-z)/\rho
\end{equation}

$\nu$要最大化,使用梯度上升法。

\subsubsection{参数设置}

ADMM 1D TV Denosing就两个超参数 $\lambda$ 和 $\rho$ ,参数设置基本不会影响大致结果,但是要注意的是$\rho$不要小于1,否则可能会造成x,z更新之后过大,出现INF。

\printbibliography

\end{document}
